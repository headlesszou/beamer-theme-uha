\documentclass[8pt]{beamer}

\usetheme{uha}

\usepackage[french,english]{babel}
\usepackage[T1]{fontenc}
\usepackage[utf8]{inputenc}
\usepackage{amsmath}
\usepackage{amsfonts}
\usepackage{amssymb}

\title{Titre de la présentation}
\subtitle{Sous-titre}
\date{Date d'aujourd'hui}
\author{Auteur de la présentation}
\institute{Université de Haute Alsace}

\begin{document}

% Title page 
\begin{frame}[plain, noframenumbering]
	\titlepage
\end{frame}

\begin{frame}[fragile]{beamer UHA}
	
	Le thème beamer uha s'inspire du thème Metropolis qui lui même s'inspire du thème HSRM de Benjamin Weiss.

	Le thème se charge de la manière suivante :

	\begin{verbatim}
	\documentclass[8pt]{beamer}
	\usetheme{uha}
	\end{verbatim}
\end{frame}


\begin{frame}{Les listes}
	Une liste
	\begin{itemize}
		\item Premier
		\item Second
		\item Troisième
	\end{itemize}

	Une énumération
	\begin{enumerate}
		\item Premier
		\item Second
		\item Troisième
	\end{enumerate}

	Une description
	\begin{description}
		\item [UHA] Université de Haute Alsace
	\end{description}
\end{frame}

\begin{frame}{Les blocks}
	\begin{exampleblock}{Exemple}
		Un exemple
	\end{exampleblock}
	\begin{alertblock}{Exemple}
		Une alerte
	\end{alertblock}
	\begin{block}{Exemple}
		Un block
	\end{block}
\end{frame}
\end{document}
